\chapter*{Liverollespil i A’kastin}\addcontentsline{toc}{chapter}{Liverollespil i A'kastin}
Der findes mange former for rollespil. I A’kastin spiller vi High Fantasy. Magi, magiske væsner, guder og liv efter døden er derfor normalt her.\\
\textbf{Karakter:} Når du spiller, skal du have en karakter i form af et karakterark. Dette ark kan du selv lave via. hjemmesiden, eller du kan møde op til en spilgang og få hjælp af en arrangør ved tjek ind. Fælles er, at det skal være printet ud, eller være på din telefon, og fremvises til arrangørerne til hver spilgang.\\
Denne karakter er din rolle, ligesom i et skuespil. Forskellen på et skuespil og det vi laver i A’kastin, er at dine replikker og din skæbne ikke er bestemt på forhånd i liverollespil. Det er derfor helt op til dig, hvordan du vil udvikle din karakter.\\
\textbf{In-game og off-game:} In-game er, når du er inde i spillet, som den karakter du har lavet et karakterark til. In-game er adskilt fra off-game, for at bibeholde illusionen om, at du nu befinder dig i en anden verden. Du må derfor ikke gå off-game midt i spillet. Har du brug for en pause, er der områder, hvor du kan gå off-game.\\
\textbf{Er du ny:} Hvis du er ny til liverollespil, anbefaler vi, at du ser på fanen “For nye spillere” på hjemmesiden. Fanen ligger under “Information”.

\section*{Arrangørerne}\addcontentsline{toc}{section}{Arrangørerne}

I A’kastin er det Live Fantasy Arrangørgruppen, der arrangerer rollespillet, og det er dermed også dem, der bestemmer til spilgangene.
Følger du ikke deres anvisninger, kan det i værste fald medføre bortvisning.
Er du i tvivl om hvem, der er medlem af arrangørgruppen, kan du se det inde på vores hjemmeside.
Officiel kommunikation med arrangørerne skal foregå over mail til akastin@gmail.com.
Til spilgange får Arrangørgruppen hjælp af et supportcrew, der hjælper til med praktiske opgaver, dette supportcrew er ikke arrangører, og kan derfor ikke svare på spørgsmål vedrørende spillet.

\section*{Opførsel i bakkerne}\addcontentsline{toc}{section}{Opførsel i bakkerne}
Spilområdet er et fredet naturområde, som vi låner. Fordi det er fredet, er der følgende regler, som skal overholdes:
\begin{itemize}
    \item Lyngen i bakkerne er fredet. Træd ikke på det, men brug de anviste stier.
    \item Lad være med at gøre skade på naturen. Hertil hører, at du for eksempel ikke må knække grene af træer eller fælde dem.
    \item Smid ikke dit affald i naturen. Smid det i en skraldespand.
    \item Hop ikke over hegnet. Brug lågerne der findes flere steder i området.
    \item Åben ild er forbudt i bakkerne. Dette gælder også trangia sæt.
    \item A’kastin rollespilsforening befinder sig i Rebild kommune, som er en røgfri kommune. Det er derfor forbudt at ryge på spilområdet. Ønsker man at ryge skal dette gøres på parkeringspladsen.
    \item Der er dyr i bakkerne, dette er normalt får. Disse har fortrins ret, og må ikke slås på. Hvis de løber mod dig, så er det din opgave at bakke væk.
    \item Brug de anviste toiletter, og gør rent efter dig selv (sminkning og lignende).
\end{itemize}
Overholdes disse regler ikke kan det medføre bøder.

\section*{Retningslinjer for A’kastins livearrangementer}\addcontentsline{toc}{section}{Retningslinjer for A’kastins livearrangementer}
Når du deltager i vores arrangementer, giver du samtykke til, at der bliver taget billeder og videoer, som bliver lagt op på vores hjemmeside, officielle facebookside og brugt til reklame for foreningen. Arrangørerne kan ikke drages til ansvar for billeder, som er taget eller bruges af privatpersoner. Har du indvendinger, bedes du kontakte arrangørgruppen over mail på akastin@gmail.com.