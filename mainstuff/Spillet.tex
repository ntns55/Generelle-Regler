\chapter{Regler for spillet}
\section{Tjek ind}
For at kunne deltage til spilgangene, skal du tjekkes ind. Dette foregår på parkeringspladsen, før vi laver en spilbriefing. Her vil du få udleveret in-game genstande og kan stille spørgsmål til arrangørerne.\\
Samtidig med tjek ind er der også våben- og rustningstjek. Alle våben og rustninger skal checkes før spilgangene, også selvom de er blevet godkendt til tidligere scenarier. Vi anbefaler, at man anskaffer sig
våben og rustning fra officielle rollespilsbutikker, da vi har erfaring med, at våben fra Netto, Fakta osv. ikke klarer sikkerhedschecket.\\
Har du brug for at tjekke ind udenfor normalt tidsrum, bedes du kontakte en arrangør.

\section{In-game}
Når du er in-game, spiller du din rolle og befinder dig i verdenen Kalish. Din rolle er baseret på race, profession, religion og baggrundshistorie. Din rolle har egne ambitioner og mål, som denne gerne vil føre ud i livet.\\
Konceptet med in-game er, at du spiller din rolle, så det er tydeligt, at du befinder dig i en anden verden. Det vil sige, at in-game foregår udenfor din normale hverdag; mobiltelefoner, snak om familiens kæledyr og lignende, vil vi derfor ikke opleve in-game.

\section{In-game kulisser}
\textbf{Telte:} Foreningen opstiller middelaldertelte til hver spilgang. Disse telte fungere som in-game bygninger. Du er altid velkommen til at tage dit eget middelaldertelt med og sætte op.\\
\textbf{Mure:} Mure er kun gældende som mure, hvis de opfylder ét af følgende punkter:
\begin{itemize}
    \item Den er lavet af en form for stof (f.eks. stofsider i et telt)
    \item Den er lavet af en presenning.
    \item Den er en reel mur, der er lavet af eksempelvis træplader eller andet hårdt materiale
\end{itemize}

Bunker af grene der er stablet, er derfor ikke gældende som mur, og er udelukkende en forhindring, som kan gås igennem.\\
Vælger en spiller at lave en mur, skal denne, selvom den opfylder ovenstående ét af ovenstående punkter, godkendes af en arrangør.\\
\textbf{Porte og døre:} En dør er bl.a. indgangen til en bygning og en port er en passage i en mur. Disse skal være synlige, og det skal være åbenlyst, at det er en indgang.\\
Porte og døre kan være låste, hvilket skal markeres med en in-game lås. Det er ikke tilladt, at fysisk blokere en dør eller port. For at komme forbi låsen skal du bruge en af følgende metoder:
\begin{itemize}
    \item Man har nøglen til låsen.
    \item Ved brug af evner.
    \item Man kan bruge en rambuk. Det kræver minimum 5 personer med en samlet nævekamp(Nævekamp forklares længere nede.) på 20. Det tager 2 minutter at nedbryde en dør og 10 minutter for en port. Det er ikke tilladt at ramme døren og dermed ødelægge den rigtigt. Situationen skal rollespilles/markeres for både døre og porte.
\end{itemize}

\section{In-game værdier}
Fjend er vores møntfod (opkaldt efter kejserfamilien Fjendebane i Nordland), og det er dem, du bruger til at betale for in-game genstande.\\
Der findes tre typer mønter. Små bronze trekanter er 1 Fjend værd. Sølv firkanter er 5 Fjend værd. og guld cirkler er 10 Fjend værd.\\
In-game genstande kan være alt fra mad til magiske artefakter. disse må hverken graves ned eller gemmes væk off-game. Hvis du bærer genstandene på dig, der ikke er beskyttet af en lås, skal disse ting afleveres, hvis du bliver gennemsøgt, når du er død. Her er det vigtigt at huske, at det ikke er tilladt at tage de sidste 3 Fjend på en person. Har en person ikke 3 fjend i mønter må man stjæle alle mønter. Stjæler du fra en kiste, må du tage alle genstande i kisten med dig.\\
Bærer du in-game værdier, skal du være sikker på ikke at tabe dem rundt om i skoven.

\section{Off-game}
Når du er off-game spiller du ikke længere din rolle, men er dig selv. Det vil sige, at du ikke må snakke med spillere, der er in-game eller på anden måde interagere med spillet. Som off-game kan du bruge din telefon mm., men det skal foregå udenfor spilområdet.\\
Når du er off-game, skal du lægge en hånd eller begge dine hænder på hovedet. Det er ikke tilladt at gå off-game midt i spillet. Herunder hører, at du ikke må “forsvinde” off-game for at undgå en kamp/konflikt. Du må heller ikke stoppe spillet, for at diskutere regler. Er du i tvivl om noget, kan du spørge en arrangør.

\section{Kommunikation}
Vi forventer, at der holdes en god tone mellem alle personer, der deltager i A’kastin såvel i spillet som udenfor spillet. Du må selvfølgelig gerne rollespille et skænderi in-game, hvor skældsord slynges rundt. Her er det vigtigt at huske, at moderne ord ikke er tilladte. Off-game og in-game snak må ikke blandes sammen.
\section{Karakterskrivning}
For at spille i A’kastin, skal du have en karakter. Du skal lave din karakter via. hjemmesiden. Vær opmærksom på, at nogle racer, profession og evner kræver specialansøgning. Disse ansøgninger har længere behandlingstid, så send dem ind i god tid før en spilgang.\\
Det er som udgangspunkt kun tilladt at have én karakter.\\

Det er muligt at ansøge om at spille en race, der ikke er beskrevet i reglerne, f.eks. en halvelver, varulv el.lign.\\

\section{Udklædning}
Du skal selv medbringe tøj og udklædning til spilgangene. Husk at tilpasse tøj og fodtøj efter de givne vejrforhold. Som spiller skal du forsøge at tilpasse din udklædning til din rolle, hvilket du kan læse nærmere om på hjemmesiden under  "Våben og kostumer" under fanen “Information”.\\
\textbf{Ny Spiller:} Som ny spiller kan du leje våben og kostume, inklusiv maling mm. ved tjek ind. Det koster 20 kr. for leje af våben og 30 kr. for kostume. Dette skal ligges tilbage i udstyrskassen, som vil stå i kroen.
\subsection{Våben og sikkerhed}
Du skal som udgangspunkt selv have våben med til spilgangene. Alle disse våben, pile, buer, skjolde mm. skal igennem våbentjek hver gang, du møder op til vores rollespil. Dette er en sikkerhedsforanstaltning for at sikre, at ingen kommer til skade grundet spillernes udstyr under spillet. Vi forbeholder os ret til at sige nej til et våben eller en rustning, hvis vi vurderer, at der er risiko for fare ved brugen heraf.\\
Våben har følgende kategorier:
\begin{itemize}
    \item Tårnskjold - Et skjold som går fra jord til over spillerens navle.
    \item Tohåndsvåben - Et våben som går fra jord over spillerens navle.
    \item Kniv - Et våben som går fra finger spids til albue.
\end{itemize}
\textbf{Det er ikke tilladt at stikke med våben der har en hård kerne.}\\ 
\textbf{Det er ikke tilladt at slå med skjolde.}\\

\section{XP}
XP er en beskrivelse af, hvor erfaren din karakter er. Du starter med 2 XP, og får 1 XP hver gang du møder op til en spilgang. Dette kan du bruge til at købe evner til din karakter. Du kan læse mere om XP under Professioner.

\section{NK og slåskampe}
Nævekamp fortæller, hvor stærk din karakter er. Din karakter starter med et bestemt antal ud fra hvilket race du spiller, men du kan tilkøbe dig flere igennem evner.\\
NK bruges til ubevæbnede slåskampe. Den med højeste NK vinder. Er der 3 eller flere til at kæmpe, er det gruppen med højest samlede NK, der vinder. Taber man en NK-kamp, mister man ikke LP, men besvimer, og kan ikke vågne i 10 minutter. Hvis du tager skade eller bliver rystet hårdt af en anden person i de 10 minutter, vågner du dog, men har ingen erindring om, hvad der er sket de sidste 10 minutter, før du besvimede.\\
En slåskamp skal rollespilles, for at være gældende.\\
For at starte en NK-kamp skal modstanderne kigge hinanden i øjnene og tydeligt sige “Nævekamp!”, hvorefter de diskret fortæller hinanden, hvor meget NK, de hver især har.\\
Du må trække  køller, stegepander, gryder, franskbrød og andre bonkvåben\footnote{Se bonk regler side \ref{Reg:Bonk}.} under nævekamp, hvilket vil give +2NK til den, der har trukket et eller flere af de små våben/andet.\\
NK kan bruges til at holde en person mod deres vilje, hvis en eller flere personer har et højere samlet antal NK end den, der holdes.\\
Har man 10 Nævekamp får man automatisk evnen: Bærer Person.

\section{LP og RP}
LP står for livspoint og RP står for rustningspoint. Dine LP fortæller dig, hvor mange slag din karakter kan tage, før den bliver bevidstløs. Karakteren starter med et antal LP afhængigt af, hvilken race du vælger, men det er muligt at tilkøbe sig flere igennem evner der er tilgængelige for nogle professioner.\\
\begin{tcolorbox}[colback=red!10!white, colframe=red!80!black, title=Ultra Vigtig Information]
Nogle magier, drikke, artefakter mv. kan give bonus LP.\\
Disse vil altid være de første man mister og kan ikke helbredes.\\
\textbf{Denne bonus kan aldrig være højere end dit start LP.} \\
\textit{Eksempel: Thorleif er kriger og menneske. Han har derfor 3 start LP og har købt Ekstra LP op til niveau 3. Det vil sige at han har 6 LP. Han drikker en drik der giver ham +10 LP. Han får derfor kun 6 ekstra LP, hvilket bringer ham til 12.}
\end{tcolorbox}
Dine RP fortæller dig, hvor mange slag din rustning kan holde til, før den går i stykker. Du skal bære rustningen, for at dine RP tæller. \textbf{RP beskytter ikke mod magi.}\\
Din rustning skal tjekkes ved rustningscheck ved tjek ind inden hver spilgang af en arrangør, som fortæller dig, hvor mange RP din rustning giver din karakter. Alle kan bære alt rustning, men professionerne bestemmer hvor meget max RP man kan få. Det vil sige at hvis man får 8 RP ved check ind, men kun kan få 4 fra sin profession så får man kun 4.\\  
RP er det første, du mister i kamp. Når din RP går på 0, skal den repareres af en smed.

\subsection{Naturlig helbredelse}
Alle karaktere har en naturlig helbredelsesevne, med undtagelse af når din karakter er bevidstløs. Den træder i kraft, når du er blevet helbredt, genoplivet eller blevet forbundet med evnen Førstehjælp.\\
Derefter vil man genvinde 1 LP per 10 minutter. Du kan ikke genvinde flere LP end dem, du normalt har.\\
\textit{Eksempel: Thorleif har normalt 3 LP. Under kamp falder han bevidstløs efter tre slag. Kort efter kommer Agnetha med førstehjælp. Hun forbinder Thorleif. Han har stadigvæk 0 LP, dvs. han næsten ikke kan bevæge sig, og han kan derfor ikke kæmpe. Men nu er reglerne for naturlig helbredelse trådt i kraft, og efter 10 min. vil han være på 1 LP. 10 min. efter det vil han have 2 LP. 10 min. senere - dvs. efter i alt 30 minutter - vil han have opnået fuld LP, altså 3 LP.}

\section{Kamp}
Når du kæmper, er det ikke tilladt at ramme andre spillere i hovedet, på halsen eller i skridtet, rammes disse steder, mister man ikke LP. Hvis en spiller overtræder disse regler gentagne gange, bedes du kontakte en arrangør. Slag på fingre tæller heller ikke som skade.\\
\textbf{Trommeslag:} Små, hurtige slag, også kaldet trommeslag, tæller ikke som skade. Et slag tæller først som skade, hvis et sving med våbnet starter bag hoften ved hvert slag. Dette gælder selvfølgelig ikke for stagevåben, f.eks. spyd.\\
Opstår der tvivl om kampene, skal dette ikke diskuteres in-game, da dette er off-game snak. Er der mistanke om snyd, bedes du kontakte en arrangør.
\subsection{Skade}
Der findes 4 typer skade:
\begin{itemize}
    \item \textbf{Normal skade} - Dette gives med helt almindelige våben, som de fleste spillere bruger. Pistoler skader også normal skade.
    \item \textbf{Magisk skade} - Gives med magi. Alle former for magisk skade går igennem RP. Hvilket betyder, at de rammer din karakters LP direkte.
    \item \textbf{Hellig skade} - Velsignede våben giver hellig skade, hvilket skal markeres med et grønt bånd på våbnet. Dette skalder mere på dæmoner, zombier, vampyrer mm., men med mindre du får andet fortalt, så skader hellig skade det samme som normal skade.
    \item \textbf{Gift} - Går igennem al form for rustning og rammer en karakters LP direkte.
\end{itemize}
Vi ser gerne, at folk rollespiller på deres skader, f.eks. hvis du bliver ramt i benet, halter din karakter måske, indtil den er helbredt. Rammes du af en ildkugle, kan du skrige i smerte over at blive brændt. Bliver du ramt af isslag, bliver du måske kold og langsom, fordi du fryser.

\section{Dødsregler}
Har du 0 LP, bliver du bevidstløs.\\
Når du er bevidstløs, skal du ligge ”død” på jorden, hvilket betyder, at du ikke kan tale eller røre dig. Er du bevidstløs skal du vente på, at en anden hjælper dig, eller der er gået 10 minutter. Se også tabellen nedenfor.\\
Bemærk, at du kan ikke forbinde dig selv ved brug af evnen Førstehjælp.\\
\textbf{Drab ved halsoverskæring:} For at skære halsen over på nogen skal de være forsvarsløse i en sådan grad, at de ikke kan bevæge sig f.eks. hvis de er besvimet, sover, er tilbageholdt ved brug af nævekamp eller hvis angriberen er uset. Gøres dette går ofret på 0 LP.
Vi ser gerne, at der rollespilles på gidseltagninger med f.eks. dødstrusler, da dette kan give en god oplevelse og en fed situation.
Man kan ikke skære halsen over på en spiller i kamp.\\
\textbf{Karakter drab:} Et karakter drab betyder, at din karakter dør permanent, og dermed ikke længere er en del af spillet. Arrangørgruppen kan til enhver tid tillade eller annullere et karakter drab.
\begin{longtable}{|p{0.15\textwidth}|p{0.25\textwidth}|p{0.25\textwidth}|p{0.20\textwidth}|}
    \hline\rowcolor{cerulean!80}
         & I live & Bevistløs &Karakter drab \\\hline
        \endhead
        Hvornår sker det? & Du har 1 LP eller mere. & Du har 0 LP.  & Henrettelse, hvor du har gjort opmærksom på, at der er tale om karakter drab. Er du ikke enig med karakter drabet , så rådfør dig med en arrangør.\\\hline

        Hvad skal du gøre? & Spille almindeligt. & Du skal ligge "bevidstløs" indtil du: \begin{itemize} 
            \item Får førstehjælp
            \item Helbredes af en helbreder
            \item Helbredes af magi
            \item Du har ligget alene i 10 minutter
        \end{itemize}& Din karakter er helt død, og kan derfor ikke spilles mere.\\\hline
        
        Genvinde LP & \begin{itemize}
            \item Naturlig helbredelse
            \item Magi
            \item Helbredende drikke
            \item Helbredere og læger
         \end{itemize}& \begin{itemize}
             \item Førstehjælp
             \item Magi
             \item Helbredende drikke
             \item Helbredere og læger
         \end{itemize}&
         Der er ingen muligheder for at genvinde LP.\\\hline
    
        Konsekvenser & & du ligger på jorden og hverken ser eller hører, hvad der foregår omkring dig. Du kan ikke huske de sidste 10 min., før du blev bevidstløs, samt alt der skete imens& Kontakt arrangørerne for at få lavet en ny rolle. \\\hline
\end{longtable}

\subsection{At bestjæle døde}
Når folk er døde eller bevidstløse, kan man tage deres in-game værdier. Dette kan gøres ved at gennemsøge personen i 30 sekunder. Det er tilladt at frasige sig denne gennemsøgning, hvis det ikke ønskes at andre kigger i eventuelle tasker eller punge.
Ligemeget om der er gennemsøgning eller ej, skal personen opgive alle sine in-game værdier, men kan beholde 3 fjend.\\
Alle genstande med en in-game lås på vil betragtes som en in-game ting. Det betyder, at sætter du en lås på din kiste, taske, pung mm. er det noget, der kan stjæles in-game.\\
Magibrugernes magiske bøger eller totems kan bestjæles, men husk at aflevere disse tilbage ved spilstop. Ønskes det at stjæle denne specifikke genstand skal kan man spørge mageren, hvilken genstand der er deres magibog eller totem, hvorefter de skal angive denne genstand.\\
Du må ikke tage personlige genstande. Dette indebærer off-game ting, våben, skjolde og rustning med mindre du får lov.\\
\textbf{Pistolskud må gerne stjæles, men pistoler må ikke selvom de skal skaffes in-game.}\\
Husk derfor at spørge den, der gennemsøges, om du er ved at tage private off-game ting.
Nogen gange kan må få lov til at stjæle nogle ting, som folk har haft med hjemmefra. Det kan være personlige ting som bøger, tasker, kister, bandager mm. Disse skal tilbageleveres, og det er din opgave at sørge for dette, så vær opmærksom på, hvad du tager og hvem du tager det fra.\\
Dette ansvar kan ikke gives videre til andre, og afleveres de personlige genstande ikke til egermanden ved spilstop, vil det blive anset som tyveri. Hvis du er i tvivl, om du er ved at tage en personlig genstand, eller ikke ved, om du kan nå at levere det tilbage, så lad være med at tage det.

\section{Bonk}\label{Reg:Bonk}
Når du udsættes for bonk, vil et våben blive lagt på din skulder, hvorefter der tydeligt vil blive sagt “Bonk!”. Dette våben skal være godkendt til bonk ved våbentjek.
Bonk må ikke bruges i kamp og skal udføres bagfra, men kan også udføres på en person, der er forsvarsløs i en sådan grad, at personen ikke kan bevæge sig f.eks. hvis de er besvimet, sover eller er tilbageholdt ved brug af nævekamp.\\
Bliver man bonket, mister man ikke LP, men besvimer, og kan ikke vågne i 10 minutter. Hvis du tager skade eller bliver rystet hårdt af en anden person i de 10 minutter, vågner du dog, men har ingen erindring om, hvad der er sket de sidste 10 minutter, før du besvimede.\\
Det er muligt at beskytte sig mod at blive bonket ved at bære hjelm.

\section{Indsamling af ressourcer}
Alle professioner kan indsamle ressourcer i minen og i urtedalen. \\
Når man indsamler ressourcer skal man slæbe sække frem og tilbage mellem minen/urtedalen og det relevante skilt. Når man har taget 5 turer frem \underline{og} tilbage, kan gå til den kroansvarlige i byen og få et ressource slag.

\section{Tortur}
For at torturere en person skal denne være forsvarsløs i en sådan grad, at personen ikke kan bevæge sig f.eks. hvis de bundet, paralyseret eller er tilbageholdt ved brug af nævekamp.
Du mister 1 LP hver gang du tortureres i 2 minutter, hvis der bliver brugt torturredskaber til dette, mister du 1 LP hver 30. sekund.
Du skal fortælle sandheden og svare på ét spørgsmål, så snart du ryger på 1 LP.\\
Du vil fortsætte med at miste LP på ovenstående måde indtil torturen stoppes eller du kommer på 0 LP og bliver bevidstløs. Vi ser gerne at denne situation rollespilles, for at give en fed situation og et godt spil.

\section{Buer og pistoler}
Alle kan bruge en bue in-game.\\
\textbf{Pile giver 2 skade.}\\
For at bruge pistol, skal du have godkendelse via. en specialeansøgning.\\ 
\textbf{En pistol giver 3 skade og vælter personen der bliver skudt omkuld.}

\section{Låse}
Du kan anskaffe dig en lås in-game for at beskytte dine in-game værdier. Du kan åbne en lås du ejer, som om du havde en nøgle. Disse låses findes i 3 niveauer.\\
Låse i niveau 1 og 2 kan også sættes på døre, port, kister, tasker og punge.\\
Låse i niveau 3 kan kun sættes på en port, dør eller kiste, og der kan sættes en ubegrænset mængde af disse herpå.\\
En ting kan kun have 1 lås på sig. Det er ikke tilladt at have kister inde i kister eller punge inde i punge der alle har låse.\\
For at komme forbi en lås du ikke ejer skal man dirke låsen op, hvilket kræver en proffessionsspecifik evne.\\
Så snart en lås sættes på en genstand, vil denne blive betragtet som en in-game værdi, hvilket betyder, at genstanden nu kan stjæles.

\section{Generelt om Magi}
Der findes magibrugere i A’kastins live scenarier. Det kræver en specialeansøgning at kaste magi.\\
For man lov til at kunne kaste magi, vil det ikke være muligt at benytte runegenstande, da magi og runer er modsætninger af hinanden.\\
Når en magi kastes, er det magikasterens ansvar at sige effekten, uanset om det er en kampmagi eller ej.\\ 
\textbf{Bliver magiens effekt eller tidsbegrænsning ikke sagt, har magien ingen effekt.}\\
Alt magisk skade går igennem RP og skjold, hvilket betyder at det rammer dine LP direkte.\\
\textit{Eksempel: Thorleif er kriger og har 3 LP og 4 RP. Han bliver ramt af magien “Lyn”, som skader 3. Da dette går igennem Peters RP, går Peter direkte på stadiet død.}\\
Magier skal siges klart og tydeligt, og der skal ikke være tvivl om hvem, der rammes.\\

\section{Ofte forekommende effekter i spillet}
\textbf{Sygdom:} En sygdom kan være naturlig forekommet eller være magisk fremstillet. Bliver du ramt af en sygdom, vil eventuel skade der medfølger denne, gå direkte på dit LP og dens effekt vil ikke forsvinde skulle du blive bevidstløs.\\
\textbf{Gift:} Dette kan komme fra f.eks. en alkemisk eliksir og eventuel skade der medfølger denne, vil gå direkte på dit LP. Skulle du blive bevidstløs vil denne effekt forsvinde.\\
\textbf{Svaghed:} Så længe du er påvirket af denne effekt vil du ikke kunne kæmpe eller løbe.\\
\textbf{Smerte:} Denne effekt fremkalder ulidelig smerte. Du skal derfor spille på at du er ramt af disse smerte i effektens længde. Denne længde skal specificeres af den person der fremkalder effekten.\\
\textbf{Paralyse:} Du vil under påvirkelse af denne effekt blive lammet og derved ude af stand til at bevæge dig. Skulle du blive bevidstløs vil denne effekt fortage.\\
\textbf{Søvn:} Denne effekt kan f.eks. komme fra en alkemisk eliksir eller magi, og vil bringe dig i en dyb søvn. Du vil kunne vægges.\\
\textbf{Ærefrygt:} Hvis man bliver ramt af Ærefrygt vil man skulle knæle.\\