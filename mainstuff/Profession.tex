\chapter{Professioner}
\section{Generelt}
Uanset race, er du nødt til at have en profession, som viser din levevej i A’kastin. Hver profession er blevet givet deres eget regelsæt således, at alle deres regler er samlet ét sted. Dette inkludere en liste over generelle evner, hvilket er evner der ikke er afhængig af din karakters profession. Her står ligeledes beskrevet, hvor meget RP professionen maks kan have, samt specielle regler der kun gælder for den specifikke profession. Her nedenfor ses en liste over, hvilke professioner der er at vælge imellem. \\

\begin{prof*}[Alkymist]
Alkymisten er ikke særlig god i kamp, men kan lave drikke til at helbrede og forgifte. De kan også forstærke dem selv eller deres kammerater med disse drikke. Hvis man går i kamp mod en alkymist alene så skal man passe på, da et enkelt stik fra en erfaren alkymist kan betyde den visse død.\\
Du kan finde mere information på Alkymisten i Alkymist sættet.
\end{prof*}

\begin{prof*}[Barde]
Barden bruger sin tid på at hygge sig. De synger sange, læser i gamle tekster og samler rygter. De er dog ikke særligt gode til at slås, men derfor kan de stadigt være gode i kamp. Deres evne til at inspirere deres allierede er kun overgået af deres evne til at sprede rygter om deres fjender og deres evne til at skrive magiske skriftruller.\\
Du kan finde mere information på Barden i Barde sættet.
\end{prof*}

\begin{prof*}[Handelsmanden]
Penge får verden til at dreje rundt. Handelsmanden ved dette bedre end alle andre. De kan potentielt skaffe alle resourcer i hele spillet, men dette kræver penge og forbindelser. De er ikke beregnet til kamp, men som den eneste profession som kan købe pistoler så skal man stadig passe på omkring dem.\\
Du kan finde mere information på Handelsmanden i Handelsmands sættet.
\end{prof*}

\begin{prof*}[Helbreder]
Når krigerne falder eller en sygdom spreder sig i landet, så er det helbrederen man har brug for. Ikke nok med at de kan helbreder folk utrolig hurtigt, så er de også i stand til at bruge urter til at helbrede med for at deres patienter kommer hurtigere ind i kampen igen.\\
Du kan finde mere information på Helbrederen i Helbreder sættet.
\end{prof*}

\begin{prof*}[Kriger]
En kriger er den bedste til at slås. Deres evne med et sværd er utrolige. Det er dem som bestemmer hvem der vinder i en kamp. De kan holde til flere slag end den almindelige person, og de kan altid finde arbejde.\\
Du kan finde mere information på krigeren i Kriger sættet.
\end{prof*}


\begin{prof*}[Smed]
En smed er ikke lige så god til at slås som en kriger, men de kan reparere rustning, lave låse, samt forbedre rustning ved at smede runer ind i den. De er de eneste som kan reparere rustning effektivt og er derfor meget eftertragtede af krigere der bærer tung rustning.\\
Du kan finde mere information på smeden i Smede sættet.
\end{prof*}

\begin{prof*}[Strateg]
En Strateg er ofte midter punktet for en organisation. De har ikke særlig meget magt selv, og vil i stedet finde at deres evner kommer fra de folk der vælger at følge dem. De kan også interagere med verdenen uden for A'kastin nemmere end andre karaktere.\\
Du kan finde mere information på strategen i Strateg sættet.
\end{prof*}


\begin{prof*}[Tyv]
En tyv kan stjæle mønter og genstande fra andre spillere. De kan stjæle næsten alt uden at folk opdager det. De kan også snigemyrde selv den stærkeste kriger.\\
Du kan finde mere information på tyven i Tyve sættet.
\end{prof*}


{\large \textbf{Alle de nedenstående professioner kræver, at der sendes en specialansøgning til arrangørgruppen.}}

\begin{prof*}[Druide - Kræver specialansøgning]
Druiden er en magikaster som beskytter naturen, og bruger den til at bringe balance til Kalish.\\
Lige som naturen skifter årstid, så skifter druidens evner og magier også fra spilgang til spilgang.\\
Du kan finde mere information på druiden i Druide sættet.
\end{prof*}

\begin{prof*}[Magus - Kræver specialansøgning]
Magus har fået deres magi på usædvanlige måder. Det er sjældent ens fra Magus til Magus. Om det er fra et magisk artefakt eller fordi deres forældre har være magiske væsner. Det betyder også at de ikke er bundet af magi på samme måde, og kan bære rustning lettere, men får ikke nær så mange magier som andre magikastere.\\
Magus er designet til spillere der ikke har brugt magi før. Det er en profession der er relativt overskuelig. Deres fokus er at kaste mange magier fra et lille udvalg.\\
Du kan finde mere information på magusen i Magus sættet.
\end{prof*}

\begin{prof*}[Præst - Kræver specialansøgning]
Præsten hjælper sine krigere i kamp. De er ikke særlig stærke når de er alene. På trods af det er de meget varierede, da hver gud giver deres egen form for magi som kan benyttes. To præster fra forskellige guder kan derfor kaste forskellig magi.\\
For at kunne spille præst, skal du vide hvilken gud du vil tilhøre. Du kan kun skifte religion hvis du får lov til det af en arrangør når du har denne rolle.\\
Du kan finde mere information på præsten i Præste sættet.
\end{prof*}

\begin{prof*}[Tempelkriger - Kræver specialansøgning]
Som tempelkriger er du den som sørger for at din guds ord bliver lov. En tempelkriger er god til at kæmpe, og kan bærer runegenstande på trods af at de også kan kaste magi. Tilgengæld har de ikke nær så mange magier og mana som en præst.\\
Man kan også være tempelkriger for naturen, og agere som skovfoged.\\
Du kan finde mere information på tempelkrigeren i Tempelkriger sættet.
\end{prof*}

\begin{prof*}[Troldmand - Kræver specialansøgning]
Troldmanden er en varieret magikaster. Nogen er stærke i kamp og kan dræbe folk med en enkelt magi. Andre er snedige og kan få folk til at glemme alt. Uanset hvad, så er en troldmand aldrig særlig holdbar. De har aldrig særlig meget liv og kan som regel dræbes med bue og pil eller en pistol. Der findes flere typer troldmand og du skal vide hvilken type du vil være når du ansøger.\\
\textbf{Dæmonlog} - En dæmonolog bruger deres evner til at kontakte dæmoner. De bruger denne magt til at få deres magier til at vare i meget lang tid.\\
\textbf{Elementalist} - En kamp troldmand. De kan kaste med ildkugler og få folk til at vende sig. Deres magier er meget effektive i kamp.\\
\textbf{Mentalist} - En troldmand der fokusere på at fordreje sind. Nogen vil mene at de kun er brugbare til at forhøre folk, men de har ikke oplevet hvordan en mentalist kan få deres fjender til at dræbe hinanden.\\
\textbf{Nekromantiker} - Nekromantikeren kan manipulere sygdomme og livsenergi. De kan lave zombier og spreder døden hvor end de går.\\
Du kan finde mere information på troldmanden i Troldmands sættet.
\end{prof*}

Når man har valgt en profession eller multiclasser til en ny, får man automatisk det første professionsniveau. Ens professionsniveau skrives på ens karakterark som en evne. For at stige i niveau skal man have brugt et bestemt antal XP inden for den profession. Dette XP er det samme uanset profession.
Det antal XP man skal bruge ses i tabellen nedenfor.

\begin{table}[!htbp]
    \centering
    \begin{tabular}{|p{0.35\textwidth}|p{0.35\textwidth}|}
    \rowcolor{cerulean!80}\hline
        Professionens niveau & Krav \\\hline

        Niveau 1 & Ingen \\\hline
        Niveau 2 & 4 XP brugt \\\hline
        Niveau 3 & 8 XP brugt \\\hline
        Niveau 4 & 16 XP brugt \\\hline
    \end{tabular}
\end{table}

Det er muligt at have mere end én profession, hvilket kaldes multiclassing, men dette kræver, at der laves en ansøgning, der skal sendes til arrangørene, og at den profession du gerne vil have adgang til, er relevant for din karakter.
Ved multiclassing vil du blive instrueret i, hvordan du skal forholde dig til modstridende evner og beskrivelser.


\section{Evner}
Når du køber evner skal du ikke lære dem. Du får dem i stedet givet. Vi vil dog meget gerne opfordre til at man rollespiller at man lærer dem.\\
\textit{Eksempel: Hilda har købt evnen Eksta LP Niv. 1, og hun har dermed 1 ekstra LP på sin karakter. For at give noget sjovt rollespil vælger hun derfor at lære den af byvagts kaptajnen Lucia, hvilket tager form af at Hilda skal beskytte sig mod slag fra slag fra byvagts kaptajnen.}\\

Nogle evner er specielle, da de kræver, at man har en anden evne for at kunne lære dem. Dette gælder evner der henviser til et specifikt niveau. Man kan f.eks ikke købe “Ekstra LP Niveau 3” uden at have Ekstra LP Niveau 1 og 2 først.\\
Det er ikke muligt at købe den samme evne flere gange.\\
Det betyder, at man ikke kan købe “Ekstra LP Niveau 1” to gange, eksempelvis igennem to forskellige professioner.\\
Nogle evner kræver specialansøgning. En ansøgning skal sendes og godkendes før den kan tages.


\section{Multiclassing}
Det er muligt at have mere end en profession. Hvis du ønsker at have flere, skal du først og fremmest sende en ansøgning for dette.\\
Hvis du er helbreder, og multiclasser til kriger, får du adgang til evner og fordele i krigerstien. Du kan dog ikke længere fortsætte med at købe evner fra helbrederstien, så derfor er det vigtigt, at du er sikker i din sag, når du multiclasser.

\subsection{Begrænsninger på professioner ved multiclassing}
Din karakters samlede Maks RP vil være gennemsnittet af dine professioners maks (Rundet ned).\\
\textit{Eksempel: Hvis du har en profession med maks RP på 0 og en profession med maks RP på 6, vil din maks RP være 3.}\\
Du kan ikke multiclasse til den samme profession, eller professioner din karakter har haft før.
