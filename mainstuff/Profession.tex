\chapter{Professioner}
\section{Generelt}
Uanset race, er du nødt til at have en profession, som viser din levevej i A’kastin. Hver profession er blevet givet deres eget regelsæt således, at alle deres regler er samlet ét sted. Dette inkludere en liste over generelle evner, hvilket er evner der ikke er afhængig af din karakters profession. Her står ligeledes beskrevet, hvilke våben og rustningstyper de kan bruge, samt specielle regler der kun gælder for den specifikke profession. Her nedenfor ses en liste over, hvilke professioner der er at vælge imellem. \\

\begin{prof*}[Alkymist]
Alkymisten er ikke særlig god i kamp, men kan lave drikke til at helbrede og forgifte. De kan også forstærke dem selv eller deres kammerater med disse drikke. Hvis man går i kamp mod en alkymist alene så skal man passe på, da et enkelt stik fra en erfaren alkymist kan betyde den visse død.\\
Du kan finde mere information på Alkymisten i Alkymist sættet.
\end{prof*}

\begin{prof*}[Barde]
Barden bruger sin tid på at hygge sig. De synger sange, læser i gamle tekster og samler rygter. De er dog ikke særligt gode til at slås, men derfor kan de stadigt være gode i kamp. Deres evne til at inspirere deres allierede er kun overgået af deres evne til at sprede rygter om deres fjender og deres evne til at skrive magiske skriftruller.\\
Du kan finde mere information på Barden i Barde sættet.
\end{prof*}

\begin{prof*}[Handelsmanden]
Penge får verden til at dreje rundt. Handelsmanden ved dette bedre end alle andre. De kan potentielt skaffe alle resourcer i hele spillet, men dette kræver penge og forbindelser. De er ikke beregnet til kamp, men som den eneste profession som kan købe pistoler så skal man stadig passe på omkring dem.\\
Du kan finde mere information på Handelsmanden i Handelsmands sættet.
\end{prof*}

\begin{prof*}[Helbreder]
Når krigerne falder eller en sygdom spreder sig i landet, så er det helbrederen man har brug for. Ikke nok med at de kan helbreder folk utrolig hurtigt, så er de også i stand til at bruge urter til at helbrede med for at deres patienter kommer hurtigere ind i kampen igen.\\
Du kan finde mere information på Helbrederen i Helbreder sættet.
\end{prof*}

\begin{prof*}[Kriger]
En kriger er den bedste til at slås. Deres evne med et sværd er utrolige. Det er dem som bestemmer hvem der vinder i en kamp. De kan holde til flere slag end den almindelige person, og de kan altid finde arbejde.\\
Du kan finde mere information på krigeren i Kriger sættet.
\end{prof*}

\begin{prof*}[Politiker]
En politiker er ofte midter punktet for en organisation. De har ikke særlig meget magt selv, og vil i stedet finde at deres evner kommer fra de folk der vælger at følge dem. De kan også interagere med verdenen uden for A'kastin nemmere end andre karaktere.\\
Du kan finde mere information på politikeren i Politiker sættet.
\end{prof*}

\begin{prof*}[Skovfoged]
Der er folk som lever i harmoni med naturen. Skovfogeden ved at byen kan være farlig, og ved at udforske skoven så vil de være i stand til at bruge de ressourcer som kan findes mere effektivt. De er også gode til at slås, men ikke helt lige så gode som en Kriger.\\
Du kan finde mere information på skovfoged i Skovfoged sættet.
\end{prof*}

\begin{prof*}[Smed]
En smed er ikke lige så god til at slås som en kriger, men de kan reparere rustning, lave låse, samt forbedre rustning ved at smede runer ind i den, eller fjerne svagheder med Mitril. De er de eneste som kan reparere rustning effektivt og er derfor meget eftertragtede af krigere der bærer tung rustning.\\
Du kan finde mere information på smeden i Smede sættet.
\end{prof*}

\begin{prof*}[Tyv]
En tyv kan stjæle mønter og genstande fra andre spillere. De må ikke tage de sidste tre mønter fra en anden spiller eller genstande så som våben. De kan dog stjæle næsten alt uden at folk opdager det. På trods af at de kun kan bruge bue og kniv, så kan de også snigemyrde selv den stærkeste kriger.\\
Du kan finde mere information på tyven i Tyve sættet.
\end{prof*}
\todo{Indsæt Politikeren}

{\large \textbf{Alle de nedenstående professioner kræver, at der sendes en specialansøgning til arrangørgruppen. Hvordan denne skal skrives og hvad den skal indeholde beskrives i punktet "Karakterskrivning".}}

\begin{prof*}[Druide - Kræver specialansøgning]
Druider kaster magi for at beskytte naturen. Der findes to typer druider. Livsdruider prøver at skabe harmoni, mens Kaosdruider prøver at dræbe alle der skader naturen. En druide kan være utrolig magtfuld, men har sjældent nogen venner til at hjælpe dem.\\
Når du ansøger skal du skrive hvilken type druide du vil ansøge om at spille. Man kan ikke spille begge typer druide, da de fundamentalt strider mod hinanden.\\
Du kan finde mere information på druiden i Druide sættet.
\end{prof*}

\begin{prof*}[Magus - Kræver specialansøgning]
Magus har fået deres magi på usædvanlige måder. Det er sjældent ens fra Magus til Magus. Om det er fra et magisk artefakt eller fordi deres forældre har være magiske væsner. Det betyder også at de ikke er bundet af magi på samme måde, og kan bære rustning lettere, men får ikke nær så mange magier som andre magikastere.\\
Magus er designet til spillere der ikke har brugt magi før. Det er en profession der er relativt overskuelig. Deres fokus er at kaste mange magier fra et lille udvalg.\\
Du kan finde mere information på magusen i Magus sættet.
\end{prof*}

\begin{prof*}[Præst - Kræver specialansøgning]
Præsten hjælper sine krigere i kamp. De er ikke særlig stærke når de er alene. På trods af det er de meget varierede, da hver gud giver deres egen form for magi som kan benyttes. To præster fra forskellige guder kan derfor kaste forskellig magi.\\
For at kunne spille præst, skal du vide hvilken gud du vil tilhøre. Du kan kun skifte religion hvis du får lov til det af en arrangør når du har denne rolle.\\
Du kan finde mere information på præsten i Præste sættet.
\end{prof*}

\begin{prof*}[Shaman - Kræver specialansøgning]
Shamanen er den eneste magikaster som ikke kan kaste magi når de er i kamp. For at gøre op for dette har de mere magtfulde magier. En shaman kan stoppes ved bruge nævekamp eller slå dem med et våben. Man skal dog passe på, da en af de stier som shamanen kan vælge giver dem lov til at bruge skjolde, rustning og våben. Kombineret med deres magier kan de have lige så meget liv som en kriger. Deres magi afhænger blandt andet af hvor mange folk de kan få til at tilbede ånder.\\
Denne profession er god til hvis man er gruppe fra en anden klub. De kræver ikke noget viden om guder, samt bliver stærkere afhængig af hvor mange folk man har med.\\
Du kan finde mere information på shamanen i Shaman sættet.
\end{prof*}

\begin{prof*}[Troldmand - Kræver specialansøgning]
Troldmanden er en varieret magikaster. Nogen er stærke i kamp og kan dræbe folk med en enkelt magi. Andre er snedige og kan få folk til at glemme alt. Uanset hvad, så er en troldmand aldrig særlig holdbar. De har aldrig særlig meget liv og kan som regel dræbes med bue og pil eller en pistol. Der findes flere typer troldmand og du skal vide hvilken type du vil være når du ansøger.\\
\textbf{Dæmonlog} - En dæmonolog bruger deres evner til at kontakte dæmoner. De bruger denne magt til at få deres magier til at vare i meget lang tid.\\
\textbf{Elementalist} - En kamp troldmand. De kan kaste med ildkugler og lave jordskælv. Deres magier er meget effektive i kamp.\\
\textbf{Mentalist} - En troldmand er fokusere på at fordreje sind. Nogen vil mene at de kun er brugbare til at forhøre folk, men de har ikke oplevet hvordan en mentalist kan få deres fjender til at dræbe hinanden.\\
\textbf{Nekromantiker} - Nekromantikeren kan manipulere med sygdomme og livsenergi. De kan lave zombier og spreder døden hvor end de går.\\
Du kan finde mere information på troldmanden i Troldmands sættet.
\end{prof*}

Når man har valgt en profession eller multiclasser til en ny, får man automatisk det første professionsniveau. Ens professionsniveau skrives på ens karakterark som en evne. For at stige i niveau skal man have brugt et bestemt antal XP inden for den profession. Dette XP er det samme uanset profession.
Det antal XP man skal bruge ses i tabellen nedenfor.

\begin{table}[!htbp]
    \centering
    \begin{tabular}{|p{0.35\textwidth}|p{0.35\textwidth}|}
    \rowcolor{cerulean!80}\hline
        Professionens niveau & Krav \\\hline

        Niveau 1 & Ingen \\\hline
        Niveau 2 & 4 XP brugt \\\hline
        Niveau 3 & 8 XP brugt \\\hline
        Niveau 4 & 16 XP brugt \\\hline
    \end{tabular}
\end{table}

Det er muligt at have mere end én profession, hvilket kaldes multiclassing, men dette kræver, at der laves en ansøgning, der skal sendes til arrangørene, og at den profession du gerne vil have adgang til, er relevant for din karakter.
Ved multiclassing vil du blive instrueret i, hvordan du skal forholde dig til modstridende evner og beskrivelser.


\section{XP}
XP er en værdi, der viser, hvor erfaren din karakter er, og hvor lang tid den har været med i spillet. Du modtager 2 start-XP, når du laver en ny karakter, og vil efter hver spilgang, hvor du har deltaget, modtage 1 XP.\\ 
Du bruger XP til at købe evner til din karakter, hvilket kan gøres imellem spilgange, altså off-game.

\subsection{At få nye evner}
Når du køber evner skal du ikke lære dem. Du får dem i stedet givet. Vi vil dog meget gerne opfordre til at man rollespiller at man lærer dem.\\
\textit{Eksempel: Hilda har købt evnen Eksta LP Niv. 1, og hun har dermed 1 ekstra LP på sin karakter. For at give noget sjovt rollespil vælger hun derfor at lære den af byvagts kaptajnen Lucia, hvilket tager form af at Hilda skal beskytte sig mod slag fra slag fra byvagts kaptajnen.}\\

\section{Generelt om evner}
Nogle evner er specielle, da de kræver, at man har en anden evne for at kunne lære dem. Dette gælder evner der henviser til et specifikt niveau. Man kan f.eks ikke købe “Ekstra LP Niveau 3” uden at have Ekstra LP Niveau 1 og 2 først.\\
Det er ikke muligt at købe den samme evne flere gange.\\
Det betyder, at man ikke kan købe “Ekstra LP Niveau 1” to gange, eksempelvis igennem to forskellige professioner.

\section{Multiclassing}
Det er muligt at have mere end en profession. Hvis du ønsker at have flere, skal du først og fremmest have det antal XP, som det kræver. Dit valg skal også godkendes af arrangørgruppen igennem en specialansøgning.\\
Hvis du er helbreder, og multiclasser til kriger, får du adgang til evner og fordele i krigerstien. Du kan dog ikke længere fortsætte med at købe evner fra helbrederstien, så derfor er det vigtigt, at du er sikker i din sag, når du multiclasser.

\subsection{Begrænsninger på professioner}\todo{delete}
Der er dog stadig begrænsninger på multiclassing. En druide, som multiclasser til en kriger må f.eks. stadig ikke bære rustning af metal, ligeledes vil en troldmand der gør det samme heller ikke kunne kaste magi, når de bærer en rustning en troldmand ikke ville kunne bære normalt. En tyv der bære pladerustning, fordi han er multiclasset præst og har taget tempelkriger, vil ikke kunne stjæle, ovs.\\
Denne regel kan forkortes til følgende:\\
\textbf{Du kan ikke bruge evner fra en profession, hvis du bære noget den profession ikke kan bruge.}\\

\subsection{XP Pris}\todo{delete?}
Det koster altid 2 XP at multiclasse, uanset hvad du multiclasser fra og til. Når du har multiclasset starter du på 0 XP i din nye profession. Tidligere brugt XP sættes også til 0.\\
\textit{F.eks. Torben er niveau 3 kriger og vil gerne multiclasse til Shaman. Han har opsparet 3 XP og har brugt 15 XP. Herefter sender han en ansøgning om både at blive Shaman og om at multiclasse til Shaman. Han bliver godkendt til dem begge, han har derfor nu 0 XP opsparet og 0 XP brugt. Han starter derfor i niveau 1 og har kun adgang til niveau 1 som Shaman. Tidligere købte evner tæller altså ikke med i mængden af XP han har brugt som Shaman.}
